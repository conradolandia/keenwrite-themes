% %%%%%%%%%%%%%%%%%%%%%%%%%%%%%%%%%%%%%%%%%%%%%%%%%%%%%%%%%%%%%%%%%%%%%%%%%%% 
%
% Configure ornamental first letter.
%
% %%%%%%%%%%%%%%%%%%%%%%%%%%%%%%%%%%%%%%%%%%%%%%%%%%%%%%%%%%%%%%%%%%%%%%%%%%% 

% Ensures that the second paragraph does not overlap the lettrine. 
% Courtesy of Max Chernoff from the ConTeXt mailing list, who writes:
%
%   "Normally manipulating nodes inside the callbacks is the wrong way to
%    do something, but in this case, that's exactly how the standard initial
%    code works. 
% 
%   "We need to modify the 'default' alternative since `\setupinitial`
%    provides no way for us to set a different alternative. A better solution
%    would be to provide a new alternative so we wouldn't need to make
%    questionable overrides to the base code."
%
\startluacode
  userdata.hangindent = 0

  function userdata.post_lettrine( head )
    nodes.tasks.disableaction( "finalizers", "userdata.post_lettrine" )

    if tex.prevgraf < math.abs( tex.hangafter ) then
      userdata.hangindent = tex.hangindent
      nodes.tasks.enableaction( "processors", "userdata.next_par" )
    end

    return head
  end

  nodes.tasks.appendaction( "finalizers", "before", "userdata.post_lettrine" )
  nodes.tasks.disableaction( "finalizers", "userdata.post_lettrine" )

  function userdata.next_par( head )
    nodes.tasks.disableaction( "processors", "userdata.next_par" )

    if head.next.id == node.id( "glue" ) and
      head.next.subtype == 20
    then
      head.next.width = userdata.hangindent
    end

    return head
  end

  nodes.tasks.appendaction( "processors", "before", "userdata.next_par" )
  nodes.tasks.disableaction( "processors", "userdata.next_par" )

  local default = typesetters.initials.actions.default

  function typesetters.initials.actions.default( ... )
    nodes.tasks.enableaction( "finalizers", "userdata.post_lettrine" )
    return default( ... )
  end
\stopluacode

\setupinitial[
  n=2,
  style=TextFontChapter,
]

